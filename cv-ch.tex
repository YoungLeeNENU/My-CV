% 李暘的中文简历
% 使用 tccv 模版
% 用 xelatex 编译生成 pdf 文件
\documentclass{tccv}
\usepackage[english]{babel}
\usepackage{graphicx,dblfloatfix,caption,afterpage}
\usepackage{fontspec,xunicode,xltxtra}
\newfontfamily\VeraSansYuanTi{Vera Sans YuanTi}
\newfontfamily\ShiShangZhongHeiJianTi{ShiShangZhongHeiJianTi}
\newfontfamily\HiraginoSansGB{Hiragino Sans GB}
\newfontfamily\ZapfChancery{ITC Zapf Chancery Std}


\begin{document}

\part{\VeraSansYuanTi 李旸\\
  \ZapfChancery Curriculum vitae}

\section{\ShiShangZhongHeiJianTi 项目和程序}

\begin{eventlist}


\item{\ShiShangZhongHeiJianTi 2011年9月 - 2011年12月}
  {\ShiShangZhongHeiJianTi 嵌入式系统设计}
  {\href{https://github.com/YoungLeeNENU/MessagePlatform}{\VeraSansYuanTi ARM9 手机短信平台}}

  \HiraginoSansGB
  使用 C 语言开发了移植有 Linux 操作系统的 ARM9 手机 \\
  短信平台中的短信发送栏和短信收件箱。\\
  上位机和下位机分别是装有 RedHat9.0 的 PC 和带有 GSM 模块的 EL-ARM9-S3C2410 开发板。\\
  短信发送栏和短信收件箱的图形界面的绘制均由 \href{http://www.minigui.org/en/}{MiniGUI} \\
  提供的图形函数完成。\\
  当有 SIM 卡插入 GSM 模块时,手机短信平台就可以和其 \\
  他手机进行短信通信。短信发送栏负责编辑并发送短信到 \\
  其他手机,短信收件箱负责显示收到的历史短信。
  

\item{\ShiShangZhongHeiJianTi 2012年5月 - 2012年7月}
  {\ShiShangZhongHeiJianTi 嵌入式系统设计}
  {\VeraSansYuanTi ARM9 的异步串口驱动程序}
  
  \HiraginoSansGB
  使用 C 语言编写了移植有 Linux 的 ARM9 的异步串口驱 \\
  动程序。\\
  目标机和宿主机分别是EL-ARM9-S3C2410 开发板和装 \\
  有 RedHat9.0(内核版本 2.4)的 PC。\\
  驱动程序中数据帧的收发和中断处理通过控制 ARM9 的 UART 接口完成。
  

\item{\ShiShangZhongHeiJianTi 2012年11月 - 2012年12月}
  {\ShiShangZhongHeiJianTi 编译器设计}
  {\href{https://github.com/YoungLeeNENU/A-samll-compiler-frontend}{\VeraSansYuanTi SNL 编译器前端}}
  
  \HiraginoSansGB
  使用 Python 编写了 SNL(一种类 Pascal 语言) 语言的 \\
  编译器前端。 \\
  完成了词法分析和语法分析,为编译器后端生成了语法分 \\
  析树。


\item{\ShiShangZhongHeiJianTi 2013年3月}
  {\ShiShangZhongHeiJianTi 网络编程}
  {\VeraSansYuanTi 校园网网络爬虫}

  \HiraginoSansGB  
  利用学校学籍表管理系统的漏洞编写了一个 Python 网络 \\
  爬虫程序下载了全校学生的照片,并且能够查询相关学生 \\
  的学籍信息。 \\
  同时也编写了一个基于 ELO 算法的 facemash 程序,对 \\
  下载的照片进行排名。


\item{\ShiShangZhongHeiJianTi 2013年5月}
  {\ShiShangZhongHeiJianTi Web 开发}
  {\href{https://github.com/YoungLeeNENU/a-vote-website}{\VeraSansYuanTi 校园投票统计网站}}

  \HiraginoSansGB    
  采用 Python 的 \href{https://www.djangoproject.com/}{Django} 框架编写了一个简易的校园投票 \\
  统计网站。用户可以在网站填写调查问卷,网站能够统计 \\
  并反馈问卷的调查结果。网站采用 MVC 开发模型,使用 MySQL 作为后台数据库。
  

\end{eventlist}

\personal
[\VeraSansYuanTi github.com/YoungLeeNENU]
{\VeraSansYuanTi 吉林省长春市净月大街 2555 号 \newline
  邮政编码 130117}
{\VeraSansYuanTi (+86) 155 6882 0995}
{\VeraSansYuanTi youngleemails@163.com}

\section{\ShiShangZhongHeiJianTi 教育水平}

\begin{yearlist}

\item[\HiraginoSansGB 计算机科学与信息技术专业 \newline
  本科学士学位]
  {\ShiShangZhongHeiJianTi 2010年 - 2014年}
  {\href{http://www.nenu.edu.cn/}{\HiraginoSansGB 东北师范大学}}
  % {\HiraginoSansGB 吉林省长春市}

\end{yearlist}

\section{\ShiShangZhongHeiJianTi 实习经历}

\begin{eventlist}
  

\item{\ShiShangZhongHeiJianTi 2013年9月 - 2013年10月}
  {\ShiShangZhongHeiJianTi 嵌入式系统设计}
  {\href{https://github.com/YoungLeeNENU/Cortex-M4-LED-Control}{\VeraSansYuanTi PC/CortexM4 的网络通讯程序}}

  \HiraginoSansGB
  作为软件开发实习生在\href{http://www.ciac.jl.cn/}{中科院长春应用化学研究所}实习。\\
  实习期间在 PC 和无操作系统的 CortexM4 上分别开发 \\
  了 TCP/UDP 的服务器/客户端。\\
  使 PC 和 CortexM4 能够互相通信并能从 PC 端控制 CortexM4 的硬件。\\
  PC 的客户端和服务器调用 Python 的 Socket 函数从指 \\ 
  定的IP地址和端口发送或接收数据。\\
  CortexM4 的客户端和服务器用 C 编写,
  使用 \href{http://savannah.nongnu.org/projects/lwip/}{LwIP} 的 Standalone 模式,指定运输层数据包的内容并采用回调 \\ 
  函数进行数据包的发送或接收。
\end{eventlist}

\section{\ShiShangZhongHeiJianTi 语言能力}

\begin{factlist}
\item{\HiraginoSansGB 汉语}{\HiraginoSansGB 作为母语具有良好的表达和沟通能力 \\
    曾参加东北师范大学留学生部对外汉语教学项目 \\ 
    并被评为优秀志愿者}
\item{\HiraginoSansGB 英语}{
    \HiraginoSansGB 大学英语四级:527分 \\
    \HiraginoSansGB 大学英语六级:462分 \\
    \HiraginoSansGB 口语流利,沟通能力强 \\
    \HiraginoSansGB 具备阅读和编写英文技术文档的能力
  }
\end{factlist}

\section{\ShiShangZhongHeiJianTi 专业技能}

\begin{factlist}

\item{\HiraginoSansGB 熟悉}{
    \HiraginoSansGB Python、C、Linux(Ubuntu/Redhat)、 \\
    Emacs、数据结构和基本算法
  }

\item{\HiraginoSansGB 了解}{
    \HiraginoSansGB ARM9、CortexM4、Keil、Perl、C++、 \\
    MySQL、git、\LaTeX、Windows、 \\
    使用 Lisp 进行函数式编程
  }

\end{factlist}

\end{document}
