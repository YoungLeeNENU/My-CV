% \title{Young Lee's two column CV}
% 
% tccv (two columns curriculum vitae) is a LaTeX class inspired by
% the template found at latextemplates.com by Alessandro Plasmati.
% 
% Create by Young Lee, the original files can be downloaded from:
% http://dev.entidi.com/p/tccv/
% 
\documentclass{tccv}
\usepackage[english]{babel}
\usepackage{graphicx,dblfloatfix,caption,afterpage}

\begin{document}

\part{Young Lee}

\section{Projects and hackings}    % TODO

\begin{eventlist}


  % 基于 ARM9 的手机短息平台
\item{September 2011 -- December 2011}
  {}
  {\href{https://github.com/YoungLeeNENU/MessagePlatform}{Text message platform for ARM9}}

  Developed the message sender and message box of a text message platform for ARM9 embedded with Linux in C. \\
  A PC with RedHat9.0 is the upper computer, and EL-ARM9-S3C2410 with extended GSM module is the lower computer. \\
  The message platform is grounded on the \href{http://www.minigui.org/en/}{MiniGUI} graphic librariy. \\
  With a SIM card placed on the GSM module, the message sender can edit and send a text message to the mobile phone with certain number,
  and the message box can show all the text messages have been recieved from other mobile phones.

  % 使用 C 语言开发了嵌入有 Linux 的 ARM9 的手机短息平台的短信发送栏和短信收件箱。
  % 上位机和下位机分别是装有 RedHat9.0 的 PC 和 EL-ARM9-S3C2410 开发板。
  % 短信发送栏和收件箱的界面都是由 MiniGUI 提供的图形绘制成的。
  % 当有 SIM 卡插入 GSM 模块时,手机短信平台就可以和其他手机进行短信通信,短信发送栏编辑并发送短息到其他手机,短信收件箱显示收到的历史短信。
  
  % ARM9 的串口驱动程序
\item{May 2012 -- July 2012}
  {}
  {Asynchronous serial port driver for ARM9}

  Programed an asynchronous serial port (followed the RS232 standard) driver for EL-ARM9-S3C2410 embedded with Linux in C. \\
  A PC with RedHat9.0 (kernel version 2.4) is the upper computer, and EL-ARM9-S3C2410 is the lower computer. \\
  The driver sends/recieves the data frames by operating the UART interfaces on EL-ARM9-S3C2410.

  % 使用 C 语言编写了嵌入有 Linux 的 EL-ARM9-S3C2410 的异步串口驱动程序。
  % 上位机和下位机分别是装有 RedHat9.0 (内核版本号 2.4) 的 PC 和 EL-ARM9-S3C2410 开发板。
  % 驱动程序中的数据帧的收发通过控制 EL-ARM9-S3C2410 的 UART 接口完成。

  % Linux Bootloader
% \item{May 2012 -- June 2012}
%   {}
%   {Bootloader for GNU/Linux}

%   Programming, installation and trial of transfer machines for assembly,
%   adjustment and testing of gas taps. 

  % SNL 编译器前端
\item{November 2012 -- December 2012}    % DONE
  {}
  {\href{https://github.com/YoungLeeNENU/A-samll-compiler-frontend}{SNL compiler's front-end}}

  Implementation of a SNL (a Pascal like language) compiler's front-end using Python. \\
  Completed the lexical analysis and syntax analysis and formed the syntax tree for the compiler's back-end.

  % 使用 Python 编写了 SNL(一种类 Pascal 语言) 语言的编译器前端。
  % 完成了词法分析和语法分析,为编译器后端生成了语法分析树。

  % 校园网网络爬虫
\item{March 2013}    % DONE
  {}
  {Web spider in campus}

  Hacked into school's students registration system
  and built a web spider in Python to download students' photoes and check their informations. \\ 
  Also built a facemash program based on the ELO Algorithm.

  % 利用学校的学籍表管理系统的漏洞编写了一个 Python 网络爬虫程序下载了全校学生的照片,并且能够查询相关学生的学籍信息。
  % 同时也利用 ELO 算法编写了一个 facemash 程序。

  % 投票网站
\item{May 2013}    % DONE
  {}
  {\href{https://github.com/YoungLeeNENU/a-vote-website}{Voting website}}

  Built a simple voting website in Python followed the MVC development pattern. \\ 
  The site is grounded on the \href{https://www.djangoproject.com/}{Django} framework
  and use MySQL as the back-end database.

  % 采用 MVC 开发模型和 Python 的 Dajngo 框架编写了一个简单的投票网站。
  % 数据库采用 MySQL。

\end{eventlist}

\personal
[github.com/YoungLeeNENU]
{No.2555, Jingyue Avenue \newline
  Changchun, Jilin \newline
  130117}
{(+86) 155 6882 0995}
{youngleemails@163.com}

% \begin{figure}[!h]
%   \includegraphics[width=3cm,height=3cm]{wechat.jpg}
% \end{figure}    

\section{Education}

\begin{yearlist}

\item[Major in Computer Science \newline
  Bachelor degree in reading]
  {2010 -- 2014}
  {\href{http://www.nenu.edu.cn/}{Northeast Normal University}}
  {Changchun, Jilin}

\end{yearlist}

\section{Internship experience}    % 实习经历

\begin{eventlist}
  
  % PC/CortexM4 的网络通讯程序
\item{September 2013 -- October 2013}
  {}
  {\href{https://github.com/YoungLeeNENU/Cortex-M4-LED-Control}{TCP/UDP Server/Client on PC and CortexM4}}

  Software development intern at \href{http://www.ciac.jl.cn/}{Changchun Institute of Applied Chemistry (CIAC)}.\\
  Developed TCP/UDP Server/Client on CortexM4 without operating system and PC. 
  Implementation of the connection and hardware control between CortexM4 and PC.\\
  PC Server/Client in Python uses socket functions to send/recieve data.\\
  CortexM4 Server/Client in C writes data into the TCP/UDP packet directly
  and sends/recieves data using callback functions provided by Standalone mode in 
  \href{http://savannah.nongnu.org/projects/lwip/}{LwIP}.
  
  % 在中科院长春应用化学研究所做软件开发实习生。\\
  % 实习期间在 PC 和无操作系统的 CortexM4 上分别开发了 TCP/UDP 的服务器/客户端,
  % 使 PC 和 CortexM4 能够互相通信并能控制 CortexM4 的硬件。\\
  % PC 的客户端和服务器使用 Python 的 Socket 函数从指定的IP地址和端口发送/接收数据。\\
  % CortexM4 的客户端/服务器用 C 编写。使用 LwIP 的 Standalone 模式,采用回调函数指定运输层每个包的内容进行发送或接收。
  
\end{eventlist}

% \begin{yearlist}

% \item{2013}
%   {ntdisp (\href{http://ntdisp.entidi.com/}{ntdisp.entidi.com})}
%   {Embedded devices programmer}

% \end{yearlist}

\section{Language skills}    % 语言能力

\begin{factlist}
\item{Chinese}{Native speaker}
  % 中文
  % 良好的中文表达能力
\item{English}{
    % 英文
    CET-4: 527\\
    % 大学英语四级:527 分
    CET-6: 462\\
    % 大学英语六级:462 分    
    Oral: good\\
    Written: good
    % 良好的英语听说能力和读写能力。
  }

\end{factlist}

\section{Software skills}    % 专业技能

\begin{factlist}

\item{Intermediate}{
    % 熟悉
    % Python, C, Linux(Ubuntu/Redhat), Emacs, 数据结构和基本算法
    Python, C, Linux(Ubuntu/Redhat), Emacs, Data Structures and Basic Algorithms
  }

\item{Basic level}{
    % 了解
    % ARM9, CortexM4, Perl, C++, MySQL,\LaTeX, git, Windows, 使用 Scheme 或其他 Lisp 方言进行函数式编程
    ARM9, CortexM4, Perl, C++, MySQL, git, \LaTeX, Windows, Functional Programming in Lisp
  }
  
\end{factlist}

\end{document}
