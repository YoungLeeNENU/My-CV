% \title{My two column CV}
% 
% tccv (two columns curriculum vitae) is a LaTeX class inspired by
% the template found at latextemplates.com by Alessandro Plasmati.
% 
% Create by Nicola Fontana, the original files can be downloaded from:
% http://dev.entidi.com/p/tccv/
% 
\documentclass{tccv}
\usepackage[english]{babel}
\usepackage{graphicx,dblfloatfix,caption,afterpage}

\begin{document}

\part{Young Lee}

\section{Projects and Hackings}

\begin{eventlist}


  % Minigui
\item{September 2011 -- December 2011}
  {}
  {\href{https://github.com/YoungLeeNENU/MessagePlatform}{Text message platform}}
  
  Upper computer: PC with RedHat9.0
  Lower computer: ARM9 with GSM module
  Software development for the industrial automation sector: configuration
  front-end in C with interface based on \href{http://www.gtk.org/}{GTK+},
  web applications and sites on LAMP platforms grounded on the
  \href{https://github.com/YoungLeeNENU/MessagePlatform}{SilverStripe} framework,
  supervisor programs in LabVIEW and remote system in
  \href{http://www.lua.org/}{Lua} on GNU/Linux systems.

  % Serial
\item{May 2012 -- July 2012}
  {}
  {Serial port's drive for ARM9}

  Designing of electrical schematics on
  2D cad and user manuals drafting.

  % Bootloader
\item{May 2012 -- June 2012}
  {}
  {Bootloader for GNU/Linux}

  Programming, installation and trial of transfer machines for assembly,
  adjustment and testing of gas taps. 

  % SNL
\item{November 2012 -- December 2012}
  {}
  {\href{https://github.com/YoungLeeNENU/A-samll-compiler-frontend}{SNL compiler's front-end}}

  Implementation of a SNL (a Pascal like language) compiler's front-end using Python.
  Complete the lexical anylasis and syntax anylasis and form the syntax tree.

  % Facemash
\item{March 2013}
  {}
  {Students photoes downloader}

  Hacked into school's students registration system
  and programmed an image downloader to download students' photoes and check infomations in Python. \\
  Also build a facemash program based on the ELO Algorithm.

  % Django
\item{May 2013}
  {}
  {\href{https://github.com/YoungLeeNENU/a-vote-website}{Voting website}}

  Build a simple voting website in Python grounded on the \href{https://www.djangoproject.com/}{Django} framework
  followed the MVC development pattern.\\
  Use MySQL as the back-end database.

  % 采用 MVC 开发模型和 Python 的 Dajngo 框架编写了一个简单的投票网站。
  % 数据库采用 MySQL。

\end{eventlist}

\personal
[github.com/YoungLeeNENU]
{No.2555, Jingyue Avenue \newline
  Changchun, Jilin \newline
  130117}
{(+86) 155 6882 0995}
{youngleemails@163.com}

% \begin{figure}[!h]
%   \includegraphics[width=3cm,height=3cm]{wechat.jpg}
% \end{figure}    

\section{Education}

\begin{yearlist}

\item[Major in Computer Science \newline
  Bachelor degree in reading]
  {2010 -- 2014}
  {Northeast Normal University}
  {Changchun, Jilin}

\end{yearlist}

\section{Internship experience}

\begin{eventlist}
  
  % TCP/UDP
\item{September 2013 -- October 2013}
  {}
  {\href{https://github.com/YoungLeeNENU/Cortex-M4-LED-Control}{TCP/UDP Server/Client on PC and CortexM4}}

  Software development intern at Changchun Institute of Applied Chemistry (CIAC).\\
  Developed TCP/UDP Server/Client on PC in Python and TCP/UDP Server/Client on CortexM4 with no operating system in C. \\
  Implementation of the connection and hardware control between PC and CortexM4.\\
  PC's Server/Client send/recieve data to assigned IP and port using the socket function.\\
  CortexM4's Server/Client directly write the data into the TCP/UDP packet using callback functions provieded by Standalone mode in LwIP.

  % 在中科院长春应用化学研究所做软件开发实习生。
  % 实习期间在PC和无操作系统的CortexM4上分别开发了TCP/UDP的服务器/客户端,
  % 使PC和CortexM4能够互相通信并能控制CortexM4的硬件。
  % PC的客户端和服务器使用Python的Socket函数从指定的IP地址和端口发送/接收数据,
  % CortexM4的客户端/服务器用C编写,使用LwIP的Standalone模式,采用回调函数指定运输层每个包的内容进行发送或接收。
  
\end{eventlist}

% \begin{yearlist}

% \item{2013}
%   {ntdisp (\href{http://ntdisp.entidi.com/}{ntdisp.entidi.com})}
%   {Embedded devices programmer}

% \end{yearlist}

\section{Language skills}

\begin{factlist}
\item{Chinese}{Native speaker}
\item{English}{
    CET-4: 527\\
    CET-6: 462\\
    Oral: good\\
    Written: good}
\end{factlist}

\section{Software Skills}

\begin{factlist}

\item{Intermediate}{
    Python, C, Linux(Ubuntu/Redhat), Emacs, Data Structures and Basic Algorithms}

\item{Basic level}{
    ARM9, CortexM4, Perl, C++, MySQL,\LaTeX, git, Windows, Functional Programming in Scheme/Common Lisp (Lisp dialect)}

\end{factlist}

\end{document}